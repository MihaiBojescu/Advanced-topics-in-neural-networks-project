\documentclass[conference]{IEEEtran}
\usepackage{cite}
\usepackage{amsmath,amssymb,amsfonts}
\usepackage{algorithmic}
\usepackage{graphicx}
\usepackage{siunitx}
\usepackage{textcomp}
\usepackage{xcolor}
\usepackage{multirow}
\usepackage{url}
\usepackage{listings}
\usepackage{epstopdf}
\usepackage{subcaption}
\usepackage[inkscapeformat=png]{svg}
\usepackage[font=small,labelfont=bf]{caption}

\epstopdfDeclareGraphicsRule{.gif}{png}{.png}{convert gif:#1 png:\OutputFile}
\AppendGraphicsExtensions{.gif}

\begin{document}

\title{Using Generative Neural Networks to generate artist-inspired artwork}

\author{
    \IEEEauthorblockN{Mihai Bojescu}
    \IEEEauthorblockA{
        \textit{Master in Artificial Intelligence and optimisation}\\
        \textit{Faculty of Computer Science}\\
        \textit{University ``Alexandru Ioan Cuza'' of Iași}\\
        Iași, Romania \\
        bojescu.mihai@gmail.com
    }
    \and
    \IEEEauthorblockN{Radu Șolcă}
    \IEEEauthorblockA{
        \textit{Master in Artificial Intelligence and optimisation}\\
        \textit{Faculty of Computer Science}\\
        \textit{University ``Alexandru Ioan Cuza'' of Iași}\\
        Iași, Romania \\
        radu.ssolca@gmail.com
    }
}
\maketitle

\begin{abstract}
    This document contains a study on how Generative Neural Networks could be applied in order to transform noise into artwork
    inspired by renoun artists. The architecture of the Network consists of a Convolutional Neural Network for the discriminator
    and a U-Net Convolutional Neural Network for the genetrator. The data on which the network is trained is a dataset with
    faimous paintings by Jean Monnet.   
\end{abstract}

\begin{IEEEkeywords}
    Generative Neural Networks, GANs, Convolutional Neural Networks, U-Nets.
\end{IEEEkeywords}

\section{Introduction}
    In this document, we will study how GANs could be used to generate pieces of art in the Jean Monnet style. GANs have shown
remarkable success in generating realistic images by using random noise as input, and we aim to use them as an application
for producing fine art.

    Monnet's paintings are characterised by their vibrant colors and expressive brushwork, and we will try to replicate this
signature in the pictures that the GAN produces, effectively creating a 'digital Monnet'. The model will play by rotation
the role of an art critic and an artist in order to improve itself until - as an art critic - it cannot distinguish between
real, pre-existing art and newly created art.

    During the building of this model, several intriguing questions were raised, ranging from the possibility of intersecting
arts with artificial intelligence, to the creativity and the originality of said results. What would the implications of using
such models in the industry of arts be? This paper also will explore the ethical implications of computer-generated craft.

\begin{thebibliography}{00}
    \bibitem{b1} I. J. Goodfellow, J. Pouget-Abadie, M. Mirza, B. Xu, D. Warde-Farley, S. Ozair, A. Courville, Y. Bengie
    (2014). ``Generative Adversarial Nets''. Département d'informatique et de recherche opérationnelle Université de Montréal
    Montréal, QC H3C 3J7. arXiv: 1406.2661v1
\end{thebibliography}

\end{document}
