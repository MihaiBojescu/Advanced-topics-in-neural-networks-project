\documentclass[conference]{IEEEtran}
\usepackage{cite}
\usepackage{amsmath,amssymb,amsfonts}
\usepackage{algorithmic}
\usepackage{graphicx}
\usepackage{siunitx}
\usepackage{textcomp}
\usepackage{xcolor}
\usepackage{multirow}
\usepackage{url}
\usepackage{listings}
\usepackage{epstopdf}
\usepackage{subcaption}
\usepackage[inkscapeformat=png]{svg}
\usepackage[font=small,labelfont=bf]{caption}

\epstopdfDeclareGraphicsRule{.gif}{png}{.png}{convert gif:#1 png:\OutputFile}
\AppendGraphicsExtensions{.gif}

\begin{document}

\title{Using Generative Neural Networks to generate artist-inspired artwork}

\author{
    \IEEEauthorblockN{Mihai Bojescu}
    \IEEEauthorblockA{
        \textit{Master in Artificial Intelligence and optimisation}\\
        \textit{Faculty of Computer Science}\\
        \textit{University ``Alexandru Ioan Cuza'' of Iași}\\
        Iași, Romania \\
        bojescu.mihai@gmail.com
    }
    \and
    \IEEEauthorblockN{Radu Șolcă}
    \IEEEauthorblockA{
        \textit{Master in Artificial Intelligence and optimisation}\\
        \textit{Faculty of Computer Science}\\
        \textit{University ``Alexandru Ioan Cuza'' of Iași}\\
        Iași, Romania \\
        radu.ssolca@gmail.com
    }
}
\maketitle

\begin{abstract}
    This document contains on how Generative Neural Networks could be applied in order to transform noise into artwork
    inspired by renoun artists.
\end{abstract}

\begin{IEEEkeywords}
    Generative Neural Networks, GANs
\end{IEEEkeywords}

\section{Introduction}
TBD

\begin{thebibliography}{00}
    \bibitem{b1} I. J. Goodfellow, J. Pouget-Abadie, M. Mirza, B. Xu, D. Warde-Farley, S. Ozair, A. Courville, Y. Bengie
    (2014). ``Generative Adversarial Nets''. Département d'informatique et de recherche opérationnelle Université de Montréal
    Montréal, QC H3C 3J7. arXiv: 1406.2661v1
\end{thebibliography}

\end{document}
